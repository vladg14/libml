\part{Contributing}
\chapter{Code contributions}
\section{General remarks}
\subsection{Getting the source code}
Go on our Savannah project page, and do a CVS checkout of the
current sources\\
\url{http://emptyurl.org} \cite{libml:savannah}

\subsection{Coding style}
80 col
\subsection{Code documentation}
Our API documentation is generated using \textit{ocamldoc}. You need to document your code using the standard \textit{ocamldoc} tags if you want your comments to be available in the code documentation.\\
Please take a look at \textit{ocamldoc}'s documentation for more information.

\subsection{Contributing to the user manual}
New features have to be documented in this book, which is made thanks to \LaTeX{}. The \textit{.tex} files are available on our CVS \cite{libml:savannah} so you get its sources when you do a checkout of our CVS. 

\section{Adding an algorithm to an already existing kind of learning object}
If you want to add an algorithm to an already existing kind of learning
object, then all you have to do is:
\begin{itemize}
\item{understand the API of the involved learning object}
\item{write a new visitor which runs your algorithm, and which will be accepted by an instance of the involved family of learning objects}
\end{itemize}

\section{Adding new learning object or family of learning objects to LibML}


\chapter{Other contributions}
\section{user manual}
This book is open to contribution. The \LaTeX{} sources are on our CVS. Feel free to send patches.

\section{Code documentation}

\chapter{Ideas which need to / could be implemented}
\begin{itemize}
\item Use the \textit{Format} module for verbosity
\item \textit{C} API
\item SWIG interfaces
\item Packages for various distributions of \gloss{GNULINUX}
\item Generate images during learnings (ex: to show an error converging to 0) or export to gnuplot or whatever
\item Lots of new learning objects!
\item Any suggestion? :-)
\end{itemize}
