\part{Getting into the API}

\chapter{Overview of the architecture}

\section{Writting a program which uses LibML}
\subsection{Concretely, what do a program using LibML looks like?}

In order to use LibML, here is what a program has to do:
\begin{itemize}
\item start the \textit{mld} daemon thanks to an \textit{exec}-like instruction. When started, the daemon listens on a port for SOAP requests which command its actions.
\item send SOAP requests to the daemon so that it performs the kind of machine learning you need and get the result.
\end{itemize}

\textit{That's all...}

\bigskip

\subsection{Which languages?}

Because of the fact that the \textit{mld} daemon uses SOAP (which is a reliable standard), LibML is usable from almost any language. We can list the following well known languages:
\begin{itemize}
\item C/C++
\item Perl
\item Java
\item Python
\item C\#
\end{itemize}
Many other languages are possible (somewhere it's said that there is a SOAP
implementation for over 60 languages...). All you need to make sure is that
you can create a SOAP server out of a \textit{.wsdl} file. The \textit{.wsdl}
file contains a description of what a given SOAP server does (\textit{mld} is
a SOAP server) and how to communicate with it. Almost all languages have
facilities for easily generating an API using the \textit{.wsdl} file.

\bigskip

The way such a thing should be implemented in the main languages is documented
in \vref{mld}.

\subsection{Basic algorithm for a developer willing to use LibML}
\begin{verbatim}
1. Find a good idea.
2. Pick a language you like to use.
3. if (This manual explains how to use LibML from this language)
   then
     Go to 4.
   else
     if (You like your language enough to handle the SOAP part,
         which is not that complicated)
     then
       Write the SOAP part for your language, and let us know.
     else
       Pick a different language and go to 3.
4. Manage to make a program which starts mld.
5. Manage the program to send a dummy SOAP instruction to the daemon.
   At this point, you have a program which can send learning tasks to
   mld, but does nothing.
6. Look at the pages of this manual which are relevant for the kind
   of implemented learning algorithm you plan to call from your
   program, and convert the given pseudo-code to the language
   you use. It's all about sending simple messages to the daemon.
7. Debug ;)
\end{verbatim}


\section{The \textit{\gloss{LEARNING}} class: where everything begins}

The learning class represents a learning session. The session is alive as
long as your instance of the \textit{\gloss{LEARNING}} class is alive. Here
are the main datas which are stored into an instance of this class.
\subsection{Code example}
The following OCaml code enables you to instanciate a \textit{learning}.
\begin{verbatim}
(* TODO: this code needs to be converted to pseudo-code *)
open Learning

let _ =
  (* This line creates an empty instance of the learning class *)
  let
    learning = new learning
  in
    exit 0
\end{verbatim}

For now, this new \textit{learning} value is filled with an unusable
learning object. You'll have to fill it with an object
who's class inherit from the \textit{\gloss{LEARNINGOBJECT}} class thanks
to its \textit{setLearningObject} method. See \vref{algorithms:part} for
more information about the available learning objects. 

\subsection{Data stored into an instance of the \textit{\gloss{LEARNING}} class}
\paragraph{An instance of the \textit{\gloss{ENV}} class}
An instance of the \textit{\gloss{ENV}} class contains the datas and
the methods used to provide some interaction (mainly to handle
verbosity).

\paragraph{An instance of a class which inherit from
the \textit{\gloss{LEARNINGOBJECT}} class}
A learning object is an abstract learning object which contains datas structures and/or methods  which enable one to perform a learning. See \vref{algorithms:part} to learn more about the available concrete learning objects.

\subsection{Useful methods of the \textit{\gloss{LEARNING}} class}
\paragraph{\textit{setLearningObject}}
In the following code, an instance of learning is filled with a usable
learning object thanks to the \textit{setLearningObject} method:
\begin{verbatim}
open Learning

let _ =
  let learning = new learning
    and learningObject = new mlpNN
  in
    learning#setLearningObject learningObject;
    exit 0
\end{verbatim}

\paragraph{\textit{getlearningObject}}
This method provides you the current instance of learningObject. The
following OCaml code provides you a function which gives you the current
learningObject. This \textit{getLearningObject} function takes one parameter:
the learning on which it is supposed to work on.
%TODO example of call to the new getLearningObject function
\begin{verbatim}
let _ =
  let getLearningObject (mylearning: learning)= (mylearning#getLearningObject)
in
  (* Then, you can use the getLearningObject function as you wish, or exit 0... *)
  exit 0
\end{verbatim} 

\section{Using visitors to perform actions}
The \textit{\gloss{DEFAULTVISITOR}} class is intensively used to perform
actions on instances of classes which inherit from the
\textit{\gloss{LEARNINGOBJECT}} class. See \vref{api:xmlimportvisitor}
for a code example about how to use a visitor to perform an action on an
object.

